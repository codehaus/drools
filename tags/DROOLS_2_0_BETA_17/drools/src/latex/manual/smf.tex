\chapter{Semantic Module Framework}
\label{smf}

\section{Introduction}

\begin{implnote}
Wait wait wait...

Reworking the SMF.  ConfigurableFoo is just crappy.
\end{implnote}

The Semantic Module Framework\index{semantic module framework} (SMF)
provides an extension point to the Drools Rule Language syntax
(see Chapter \vref{drl}).  By creating semantic modules,
domain-specific rule types, object-types, conditions, extractors and consequences
can be added to the core DRL language.

\section{\indexClass{Declaration} objects}
\label{smf.declaration}

Each \indexClass{Rule} has a set of parameters that specify what input objects
are required through the evaluation of the rule.  Each parameter is
modelled through a \indexClass{Declaration} object which specifies the
parameter's \emph{identifier} and \emph{object type}.  The
identifier of a \class{Declaration} is simply a string name that
uniquely distinguishes one parameter from another.  The object type
is defined through an \class{ObjectType} component
(Section~\vref{smf.objecttype}).  Various semantic components, notably
\indexClass{Condition} and \indexClass{Consequence} implementations
must be cognizant of the available declarations of a rule.

\section{\indexClass{Tuple} objects}

Objects flow through the conditions and consequence of a rule in the
form of a \indexClass{Tuple}.  A tuple is merely an associative
array, not unlike a \class{HashMap}, which indexes the value of
the objects by its \indexClass{Declaration}.  \class{Condition} and
\class{Consequence} implementations operate upon \class{Tuple} objects
by retrieving values by their \class{Declaration}
(Figure~\vref{smf.tuple}).

\begin{figure}
\begin{javaCodelisting}
package org.drools.spi;

public interface Tuple
\{
    ...
    Object get(Declaration declaration);
    ...
\}
\end{javaCodelisting}
\caption{Method to retrieve an object by its 
\indexClass{Declaration} from a \indexClass{Tuple}}
\label{smf.tuple}
\end{figure}



\section{Semantic components}

\subsection{\indexClass{RuleType}}

\begin{implnote}
This doesn't exist at the moment, but will soon.  \class{RuleType}
components will allow a module to be responsible for the
initialization of rule, possibly adding parameters, condition and
consequences implicitly.
\end{implnote}

Implementations of \indexClass{RuleType} allow a module to initialize
a new rule, possibly by adding parameters, conditions or consequences.
The \class{RuleType} is passed the new \indexClass{Rule} for
initialization.

\begin{figure}
\begin{javaCodelisting}
package org.drools.smf;

public interface RuleType
\{
    void initializeRule(Rule rule,
                        Configuration config)
        throws RuleTypeException;
\}
\end{javaCodelisting}
\caption{\indexClass{RuleType} interface}
\label{ruletype}
\end{figure}

\subsection{\indexClass{ObjectType}}
\label{smf.objecttype}

Implementations of \indexClass{ObjectType} allow rules to be
defined in terms of semantic types. Instead of requiring all
rules to be defined in terms of Java classes, rules may be
defined in terms of higher semantics.  For example, while
all XML documents may be instances of \texttt{org.w3c.dom.Document}
each document may have a different semantic type based upon
the name and namespace of the root tag.  Given an \texttt{Object},
an \indexClass{ObjectType} implementation must simply determine
if it matches its semantic type (Figure \vref{objecttype}).

\begin{figure}
\begin{javaCodelisting}
package org.drools.spi;

public interface ObjectType
    extends SemanticComponent
\{
    boolean matches(Object object);
\}
\end{javaCodelisting}
\caption{\indexClass{ObjectType} interface}
\label{objecttype}
\end{figure}

\subsubsection{\indexClass{ConfigurableObjectType}}

Configuration information may be passed to an \indexClass{ObjectType}
if the implementation is marked with the
\indexClass{ConfigurableObjectType} interface.
A \indexMethod{ConfigurableObjectType}{configure(...)} method
is added by the \indexClass{ConfigurableObjectType} interface.
through which additional information may be passed
(Figure~\vref{configurableobjecttype}).

\begin{figure}
\begin{javaCodelisting}
package org.drools.smf;

public interface ConfigurableObjectType
    extends ObjectType
\{
    void configure(Configuration config) throws ConfigurationException;
\}
\end{javaCodelisting}
\caption{\indexClass{ConfigurableObjectType} interface}
\label{configurableobjecttype}
\end{figure}

\subsection{\indexClass{Condition}}

Implementations of the \indexClass{Condition} interface allow for
custom conditions to be created.  Each condition is effectively
a predicate object that returns a boolean value given some input
data.  In order to insert the \indexClass{Condition} into the
appropriate location within the Rete\index{Rete} graph, each
\class{Condition} must specify the declarations of the variables
with which is analyzes (Section~\vref{smf.declaration}).

Figure~\vref{condition} show \indexClass{Condition} interface.
Any value expected to be used by the
\indexMethod{Condition}{isAllowed(...)}
method must be accounted for in the array of \indexClass{Declaration}
objects returned by \indexMethod{Condition}{getRequiredTupleMembers}.
For example, a rule may declares three object: \texttt{a}, \texttt{b},
and \texttt{c}.  A particular \class{Condition} implementation might
only test attributes of the \texttt{b} object.  In that case, the
\indexClass{Declaration} associated with the \texttt{b} object must
be returned from
\indexMethod{Condition}{getRequiredTupleMembers(...)}.

The \indexMethod{Condition}{isAllowed(...)} method is passed a
\indexClass{Tuple} when enough knowledge to satisfy the required
members is available.  It may then use the operations of \class{Tuple}
to retrieve the objects and perform its test to return a boolean.

\begin{figure}
\begin{javaCodelisting}
package org.drools.spi;

public interface Condition
    extends SemanticComponent
\{
    Declaration[] getRequiredTupleMembers();

    boolean isAllowed(Tuple tuple) throws ConditionException;
\}
\end{javaCodelisting}
\caption{\indexClass{Condition} interface}
\label{condition}
\end{figure}

\subsubsection{\indexClass{ConfigurableCondition}}

\subsection{\indexClass{Extractor}}

\subsubsection{\indexClass{ConfigurableExtractor}}

\subsection{\indexClass{Condition}}

\subsubsection{\indexClass{ConfigurableCondition}}

\subsection{\indexClass{Consequence}}

\subsubsection{\indexClass{ConfigurableConsequence}}

\section{The \indexClass{Configuration} structure}

For the \texttt{Configurable...} form of the semantic components, 
configuration information is communicated through a tree of
\indexClass{Configuration} objects.  Each \class{Configuration}
object acts as a node in the tree, and may contain the following
data:

\begin{itemize}
  \item \textbf{Attributes.} Zero or more name/value pairs of strings
(Figure~\vref{configuration.attributes}).
  \item \textbf{Text.} A single string text value
(Figure~\vref{configuration.text}).
  \item \textbf{Child \class{Configuration} nodes.} Zero or more named
    child \class{Configuration} nodes
(Figure~\vref{configuration.children}).
\end{itemize}

\begin{figure}
\begin{javaCodelisting}
String[] attrNames = config.getAttributeNames();
String   someValue = config.getAttribute( someName );
\end{javaCodelisting}
\caption{Attribute-related operations of \indexClass{Configuration}}
\label{configuration.attributes}
\end{figure}

\begin{figure}
\begin{javaCodelisting}
String nodeText = config.getText();
\end{javaCodelisting}
\caption{Text-related operation of \indexClass{Configuration}}
\label{configuration.text}
\end{figure}

\begin{figure}
\begin{javaCodelisting}
Configuration[] allChildren    = config.getChildren();
Configuration   someFirstChild = config.getChild( someName );
Configuration[] someChildren   = config.getChildren( someName );
\end{javaCodelisting}
\caption{Child-related operations of \indexClass{Configuration}}
\label{configuration.children}
\end{figure}

The tree of \indexClass{Configuration} nodes may be thought of as
a simplified version of an XML structure.  For configurable semantic
components used through the DRL (Chapter~\vref{drl}), the root 
\class{Configuration} is based upon the component's own tag, and
children tags are represented by children \class{Configuration}
nodes.

\section{Semantic Module Descriptor}
\label{module.descriptor}

Conforming semantic modules are packaged as individual JAR files which
can be added to the application's classpath.  Each JAR should contain,
in the \texttt{META-INF} directory a file named
\texttt{drools-semantics.properties} which provides meta-information about
the module and its available semantic components.

\begin{implnote}
Still working on the module descriptor format.
\end{implnote}

\section{\indexClass{SemanticsRepository}}

A \indexClass{SemanticsRepository} manages a set of
\indexClass{SemanticModule} objects and allows each to be looked-up
by its URI (Figure~\vref{semanticsrepo}).  Primary a \indexClass{SemanticsRepository} is used by
a \indexClass{RuleSetReader} (Section~\vref{admin.rules.loading})
in order to extend the core DRL (Chapter~\vref{drl}) syntax.

\begin{figure}
\begin{javaCodelisting}
SemanticsRepository repo       = locateSemanticsRepository();
SemanticModule[]    modules    = repo.getSemanticModules();
SemanticModule      someModule = repo.lookupSemanticModule( someUri );
\end{javaCodelisting}
\caption{Usage of the \indexClass{SemanticsRepository}}
\label{semanticsrepo}
\end{figure}

\section{The \indexClass{DefaultSemanticsRepository} helper}

The \indexClass{DefaultSemanticsRepository} helper class is useful
in that it contains all conforming semantic modules available on
the classpath.  Each module that has a module descriptor
(Section~\vref{module.descriptor}) located within the 
\texttt{META-INF} directory.  Each \texttt{drools-semantics.properties}
will be automatically discovered by the
\class{DefaultSemanticsRepository}
upon first use.  

Being a help class that is initialized once, it follows the
singleton pattern.  To use the
\indexClass{DefaultSemanticsRepository}, the
\indexMethod{DefaultSemanticsRepository}{getInstance()}
method will retrieve the singleton instance
(Figure~\vref{defaultsemanticsrepo}).

\begin{figure}
\begin{javaCodelisting}
SemanticsRepository repo    = DefaultSemanticsRepository.getInstance();
SemanticModule[]    modules = repo.getSemanticModules();
\end{javaCodelisting}
\caption{Retrieving and using the
\indexClass{DefaultSemanticsRepository} helper}
\label{defaultsemanticsrepo}
\end{figure}

