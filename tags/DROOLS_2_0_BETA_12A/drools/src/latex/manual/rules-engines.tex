\section{Rules Engines}

Rules engines were developed to make the creation and maintenance
of collections of rules easier and less costly.
Through the use of intelligent
algorithms (see Chapter \vref{algorithms}), some dedicated rules engine
can also produce efficiencies when working with a large amount of
rules, events and data.

A good rules engine allows the business logic of a system
to be specified external to the system itself.  No longer must these
rules be codified by the developers.  Many rules engines even provide
natural-language or wizard-style GUIs for designing rules, allowing
product managers or business analysts to actually specify the logic.
Separation of concerns and responsibility is achieved by moving the 
business rule specification outside of the actual program logic.
Developers can
concern themselves with systems engineering while the analysts 
concentrate on the business logic.

There are currently many commercial and open-source implementations
of rules engines besides Drools.  The commercial
vendors have undoubtedly good algorithms and have spent considerable
time and effort on the user interfaces and rule specification
languages.

\begin{itemize}
	\item \textbf{\textsf{ILOG JRules$^{TM}$}}\\
		 \url{http://www.ilog.com/}
	\item \textbf{\textsf{Haley Eclipse$^{TM}$}}\\
		 \url{http://www.haley.com/}
	\item \textbf{\textsf{Sandia Jess$^{TM}$}}\\
		 \url{http://herzberg.ca.sandia.gov/jess/}
	\item \textbf{\textsf{CLIPS$^{TM}$}}\\
		 \url{http://www.ghg.net/clips/CLIPS.html}
\end{itemize}


