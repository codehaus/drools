\section{Rete-OO\index{Rete-OO}}

The Rete algorithm works wonderfully in language systems 
such as Lisp where pertinent attributes about objects 
are directly asserted to the rules engine.  In an
object-oriented language, such as C++ or Java, and entire 
graph of objects can be reachable from a single named 
root object.  Expressing highly complex relationships between
entities using Cambridge-prefix notation may require many
separate assertions.  In an OO language, the single root
object is all that should be asserted, since attributes 
and relationships can be \emph{extracted} using normal
language constructs.

Bob McWhirter\index{McWhirter, Bob} of The Werken Company adapted
Forgy's original Rete\index{Rete}
algorithm to object-oriented constructs, creating the Rete-OO algorithm.
As with Rete, there are \emph{1/1} nodes and \emph{2/1}
nodes.  Unlike Rete, there are nodes that exist simply 
to extract reachable attributes and add columns to passing
tuples. Rete always constructs the condition \emph{1/1} nodes toward
the root of the tree leaving the bottom portion to be 
comprised of purely aggregating \emph{2/1} \emph{join} nodes.  
Rete-OO must interleave both \emph{1/1} and \emph{2/1} nodes.

The same example as in Section \ref{algo.rete}, 
the conditions could be expressed in terms of object-oriented
language boolean and assignment expressions.  The choice
of Java as the expression language is purely arbitrary.

\medskip

\begin{tabular}{cl}
(0) & \texttt{Person personOne, personTwo} \\
(1) & \texttt{personOne.hasSister(personTwo)} \\
(3) & \texttt{petName = personOne.getCat().getName()} \\
(3) & \texttt{petName = personTwo.getDog().getName()} \\
\end{tabular}

\bigskip

Rete-OO adds the concept of \emph{root object declaration}, where 
the root objects of the condition are declared with a
name and type.  The object's type maps directly to the
tuple type in Rete.  The root object name has no direct
mapping in Rete and causes the addition of a \emph{parameter node}
in Rete-OO.  Boolean expressions in Rete-OO conditions are
equivalent to Rete's condition patterns against attributes.
The assignment expressions map to place-holder variables in
Forgy's algorithm.

The types of nodes used in Rete-OO graph construction are
listed here.  Those that are new or different from Rete are
denoted with a '*'.

\begin{itemize}
	\item \textsf{\textbf{Object type}} \\
		Object type nodes differentiate objects by
		filtering on their defined type.
	\item \textsf{\textbf{Parameter*}}\\
		Parameter nodes create a tuple with a single
		entry binding the object to the name.
	\item \textsf{\textbf{Condition}}\\
		Condition nodes simply tests a tuple against 
		an a boolean expression.
	\item \textsf{\textbf{Extraction*}}\\
		Extraction nodes extract new attributes, 
		create new columns on tuples, and store the
		results.
	\item \textsf{\textbf{Join}}\\
		Join nodes connect the output arcs from two
		other nodes and allows consistent tuples to
		be merged and passed through.
	\item \textsf{\textbf{Terminal}}\\
		Terminal nodes fire to indicate a successful
		match for the rule.  
\end{itemize}

The resulting Rete-OO graph is constructed in a different
manner than the equivalent Rete graph, due to the addition
and rearrangement of some nodes.

\begin{figure}
\begin{center}
  \begin{minipage}{6in}
	\xymatrix @R-10pt{
		\bullet \ar[d] \\
		type(Person) \ar[d] \ar[dr] \\
		parameter(personOne) \ar[d] & parameter(personTwo) \ar[d] \\
		extraction(petName) \ar[dr] & extraction(petName) \ar[d] \\
		 & join \ar[d] \\
		 & condition(hasSister) \ar[d] \\
		 & terminal \\
	}
  \end{minipage}
\end{center}
\caption{Rete-OO network}
\label{network.rete-oo}
\end{figure}
