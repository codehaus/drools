
\chapter{Rule Assembly}

\section{Overview}

Only a handful of classes are required to assemble rules once a
\emph{semantic module} has been selected.  Each rule is codified as an
instance of the \verb|Rule| class which may be a member of a
\verb|RuleSet| collection.  

\begin{enumerate}
	\item Instantiate a \verb|Rule|.
	\item Add a \verb|Declaration| for each root fact object.
	\item Add a \verb|Extraction| for each fact extraction.
	\item Add a \verb|Condition| for each restrictive condition.
	\item Add a \verb|Consequence| for performing the result of a match.
	\item Add the \verb|Rule| to a \verb|RuleBase|.
	\item Optionally register the \verb|RuleBase| with a \verb|RuleBaseRepository|
\end{enumerate}

When adding a \verb|Rule| to a  \verb|RuleBase|, it is possible that
the rule cannot be integrated into the network.  This is caused
by \verb|Extraction| or \verb|Condition| objects that expect
\verb|Declarations| that are otherwise not present in the rule.

\newpage

\section{Rule Assembly Example}

Rules may be assembled using classes from the \verb|org.drools.rule|
package along with one or more semantic modules.  Additional tools
to allow for assembling rules from a file or database are possible.

\footnotesize
\begin{alltt}
// -- Create a new Rule

Rule rule = new Rule("example");

// -- Create the semantic Person object type
// -- which maps directly to java Person type.

ObjectType personType = new ObjectType() \{
        public boolean matches(Object object) \{ 
            return ( object instanceof Person );
        \}
    \};

// -- Create the semantic String object type.
// -- which maps directly to java String type.

ObjectType stringType = new ObjectType() \{
        public boolean matches(Object object) \{ 
            return ( object instanceof String );
        \}
    \};

// -- Declare two root fact Person objects 
// -- with the identifiers 'sisOne' and 'sisTwo'

final Declaration sisOneDecl = new Declaration( personType,
                                                "sisOne" );

final Declaration sisTwoDecl = new Declaration( personType,
                                                "sisTwo" );

// -- Declare the extracted String object
// -- with the identifier 'petName'

final Declaration petNameDecl = new Declaration( stringType,
                                                 "petName" );

// -- Add the root fact Person declarations
// -- to the rule.

rule.addParameterDeclaration( sisOneDecl );
rule.addParameterDeclaration( sisTwoDecl );

\newpage

// -- Create the fact extractor for the dog name

Extractor dogNameExtractor = new Extractor() \{
        public Declaration[] getRequiredTupleMembers() \{
            return new Declaration[] \{ sisOneDecl \};
        \}
        public Object extractFact(Tuple tuple) \{
            Person person = (Person) tuple.get( sisOneDecl );
            return person.getDog().getName();
        \}
    \}
      );

// -- Create the fact extractor for the cat name

Extractor catNameExtractor = new Extractor() \{
        public Declaration[] getRequiredTupleMembers() \{
            return new Declaration[] \{ sisTwoDecl \};
        \}
        public Object extractFact(Tuple tuple) \{
            Person person = (Person) tuple.get( sisTwoDecl );
            return person.getCat().getName();
        \}
    \}
      );

// -- Add the extractions for the dog and cat
// -- name, both to the 'petName' variable.

rule.addExtraction( new Extraction( petNameDecl,
                                    dogNameExtractor ) );

rule.addExtraction( new Extraction( petNameDecl,
                                    catNameExtractor ) );

// -- Add a filter that only allows two
// -- Persons who are sisters to pass.

rule.addCondition( new Condition() \{
        public Declaration[] getRequiredTupleMembers() \{
            return new Declaration[] \{ sisOneDecl, sisTwoDecl \};
        \}
        public boolean isAllowed(Tuple tuple) \{
            Person sisOne = (Person) tuple.get( sisOneDecl ) 
            Person sisTwo = (Person) tuple.get( sisTwoDecl ) 
            return sisOne.hasSister( sisTwo )	;
        \} 
    \}
      );

\newpage

// -- Attach an action to fire when matched.

rule.setConsequence( new Action() \{
        public void invoke(Tuple tuple, WorkingMemory memory) \{
            System.err.println( "sisOne: " + tuple.get( sisOneDecl ) );
            System.err.println( "sisTwo: " + tuple.get( sisTwoDecl ) );
            System.err.println( "petName: " + tuple.get( petNameDecl ) );
        \}
    \}
      );

// -- Create a new rule base

RuleBase ruleBase = new RuleBase();

try
\{
    // -- Add the rule to the rule-base.
    // 
    // -- May throw a ReteConstructionException 
    // -- if the Rule cannot be integrated into
    // -- the Rete-OO network.

    ruleBase.addRule( rule );
\}
catch (ReteConstructionException e)
\{
    e.printStackTrace();
    return;
\}

// -- Create a repository.

SimpleRepository repo = new SimpleRepository();

// -- Register the RuleBase with the repository.
	
repo.registerRuleBase( "http://rules.werken.com/family-relationships",
                       ruleBase );

\end{alltt}
\normalsize

\newpage
