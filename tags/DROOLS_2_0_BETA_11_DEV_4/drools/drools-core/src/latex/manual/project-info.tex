\chapter{Project Information}


\section{Web Site}

The main focal point of the drools project is the website, where
information regarding source \& binary distributions, source
repository, mailing lists and chat archives can be found:

\begin{quote}
    \url{http://drools.org/}
\end{quote}

\section{Mailing Lists}

The drools project maintains two mailing lists.  The first, known as
\verb|drools-interest| is for general discussion by users and
developers of drools.  The second list is \verb|drools-cvs| which
simply tracks changes made to the source-code through the CVS
repository. For information about subscribing to each list or access 
to the list archives:

\begin{quote}
    \url{http://lists.werken.com/listinfo/drools-interest}\\
    \url{http://lists.werken.com/listinfo/drools-cvs}
\end{quote}

\section{Source Repository}

The drools project maintains a revision control repository using
CVS.  To checkout the latest sources, you must issue two CVS commands.
The first is used to login.  When presented with a prompt for a
password, simply press \emph{ENTER}.

\begin{verbatim}
cvs -d:pserver:anonymous@cvs.werken.com:/cvsroot/drools login
cvs -d:pserver:anonymous@cvs.werken.com:/cvsroot/drools co drools
\end{verbatim}

\section{Internet Relay Chat}

There is a dedicated channel on The Werken Company's IRC server for
drools:\\

\begin{tabular}{rl}
address & \verb|irc.werken.com| \\
port    & \verb|6667| \\
channel & \verb|#drools|\\
url     & \url{irc://irc.werken.com:6667/drools}\\
\end{tabular}

\bigskip

Archives of chats in the channel are maintained on-line:

\begin{quote}
    \url{http://irc.werken.com/channels/drools/}
\end{quote}

\section{Bug \& Issue Tracking}

The Werken Company provides access to their JIRA issue tracking server
to support drools:

\begin{quote}
    \url{http://jira.werken.com/}
\end{quote}
