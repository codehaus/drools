\chapter{Functional Overview}

\section{Rules}

\subsection{Declarative Form}

Drools directly supports \emph{declarative}\index{declarative} rules,
as opposed to \emph{procedural}\index{procedural} logic.  Declarative
rules typically take the form as follows:

\begin{quote}
  \textbf{if} \emph{condition} \\
    \textbf{then} \emph{consequence}
\end{quote}

By ``declarative', it is meant that the rules \emph{declare}, by
way of the \emph{condition}, what should occur but do not specify the
procedure for actually testing the conditions. 
For example, a procedural method for ensuring you have an umbrella if
it is raining would be:

\begin{quote}
    Step outside and determine if it is raining.  If it is raining,
then go to the closet and get an umbrella.
\end{quote}

\noindent
In a declarative form, the above could be represented by two rules:

\begin{itemize}
  \item \textbf{if} \emph{it is raining} \\
    \textbf{then} \emph{you need an umbrella}
  \item \textbf{if} \emph{you need an umbrella} \\
    \textbf{then} \emph{get one from the closet}
\end{itemize}

Given declarative rules, the knowledge that it is raining could
produce two courses of action:

\begin{enumerate}
  \item You already have an umbrella, perhaps because you always carry
one, in which case, you're ready.
  \item You don't have an umbrella, so you go get one from the closet.
\end{enumerate}

\subsection{Applicable Context}

The set of rules to be considered at any point of time depend upon the
current context.  As a human, you have certain rules to think about
when you dine at a fine restaurant, which are exclusive to the set of
rules you consider when spending a sunny day at the swimming pool.
Even so, there are some other overriding rules that may be important
regardless of the context, such as laws against homocide.

So, in a given context, different sets of rules are pertinent.  A
single set of rules may be user in one context, while a different set
is used in a different context. Within a rule-engine, available sets 
of rules are called \emph{rule sets}\index{rule set}.  The set or 
sets of rules currently applicable given the context is called a 
\emph{rule base}\index{rule base}.

\section{Knowledge}

You become aware of knowledge, in the form of facts, over time.
Likewise, over time, facts may change or cease to be true, in 
which case, they facts as you know them must be altered or purged
from your memory.  Likewise, within a rule-engine, there are
operations for becoming aware of a fact, purging a fact, or
modifying a known fact.  These operations are as follow:

\begin{itemize}
  \item \textbf{Assert} Add a fact to what is known.\\
    \emph{The weather is rainy.} \emph{Betty is in the room.}
  \item \textbf{Retract} Remove a fact from what is known.\\
    Removing the fact \emph{Betty is in the room} when Betty leave the
    room.
  \item \textbf{Modify} Alter a fact from what is know.\\
    Changing the knowledge about the weather from \emph{The weather is
    rainy} to \emph{The weather is sunny} when the rain stops and the
    clouds go away.
\end{itemize}

Within a rule-engine, the collected knowledge is called the
\emph{working memory}\index{working memory}.  Knowledge is
asserted, retracted and modified within the working memory and the
rules are evaluated to determine what actions, if any, should be
taken.

\section{Why a rule-engine?}

While the logic expressed in a rule can and has often been writen
within the code a system, a rule-engine offers many benefits.  Instead
of locking the logic up in code written by developers, the logic can
be moved out-board external to the actual application.  In this way
it is possible for non-developers to change the logic without having
to rebuild the system.  Additionally, by codifying all of the system
rules in a central location, they are no longer scattered throughout
the application.  This allows for easier validation of the system's
requirements and analysis of the logic of the system.

Additionally, a rule-engine such as Drools is built upon an
intelligent algorithm that allows for the evaluation of many
rules against many facts in an effecient manner.  In a procedural
system, a change in a single fact might require double-checking
\emph{every} rule to determine if any action needs to be taken.
A rule-engine which uses the \emph{Rete}\index{Rete} algorithm
is optimized to minimize the amount of processing effort that is
required to evaluate the rules that may have been affected by
a change in knowledge.

\section{A note about ``business rules''}

A higher-level form of rule is the \emph{business rule}\index{business
rule}.  Business rules do not necessarily follow the \emph{if-then}
form, but may be specified in different formats that are not as
closely linked to the underlying rule-engine implementation.  Business
rules tend to use the \emph{must} or \emph{must not} form to
expression constraints or inferences.

\begin{itemize}
  \item An order \textbf{must not} be billed before it ships.
  \item An applicant for store credit \textbf{must} be 18 years of age.
\end{itemize}

Drools does not directly support this level of business rules, but
other projects built upon Drools\footnote{The \textbf{Fluxtapose}
suite of tools from The Werken Company is one such product that
supports business rules.} may easily support such
notation.

