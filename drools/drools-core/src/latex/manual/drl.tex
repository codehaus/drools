\chapter{Drools Rule Langauge}

\section{Introduction}

\drools{} defines a semantic-module-independant rule language created 
called the \emph{Drools Rule Language}.  DRL is a an XML-based
language that uses modern XML features such as XML-Namespaces and XML
Schema. The DRL engine within \drools{} is built on top of
jakarta-commons-jelly, which is a general XML tag library
engine. 

\section{DRL Files}

DRL files are XML files using the DRL tags. They are typically files
that have the \emph{.drl} suffix.  \drools{} only requires that they
be accessible through a URL:

\begin{itemize}
	\item \textbf{\textsf{Local Filesystem}} \\
		DRL files can be stored in the local filesystem and
		accessed using \verb|file://| URLs.
	\item \textbf{\textsf{Web/FTP Server}} \\
		DRL files can be stored on the network and
		accessed using \verb|http://| and \verb|ftp://| URLs.
	\item \textbf{\textsf{Java Classpath}} \\
		DRL files can be stored in the Java classpath or in a JAR
		file, and accessed using the \verb|getResource()| method
		on \verb|java.lang.ClassLoader| and \verb|java.lang.Class|
		classes.
\end{itemize}

\section{Loading DRL Files}

A \verb|RuleSetLoader| is provided for loading DRL files into a
\verb|RuleBase|.  Given a URL, the \verb|RuleSetLoader| will retrieve
the DRL file and load all rules and rule-sets into the specified
\verb|RuleBase| which can then immediately be used for knowledge
manipulation.

\begin{codelisting}
import org.drools.RuleBase;
import org.drools.WorkingMemory;
import org.drools.io.RuleSetLoader;
...

    RuleBase ruleBase = new RuleBase();

    RuleSetLoader loader = new RuleSetLoader();

    loader.load( rulesUrl, ruleBase );

    WorkingMemory memory = ruleBase.createWorkingMemory();

    memory.assertObject( account );
\end{codelisting}

\section{Base DRL Syntax}

The base syntax for DRL contains a small handful of tags 
representing the general structure of rules and rule-sets.
These tags are defined for the XML namespace URI of
\verb|http://drools.org/rules|.

\begin{itemize}
	\item \verb|<rules>|\\
		General outter-level wrapper tag.
	\item \verb|<rule-set>|\\
		A named collection of rules.
	\item \verb|<rule>|\\
		A single rule.
	\item \verb|<parameter>|\\
		A root fact-object parameter.
	\item \verb|<declaration>|\\
		A local variable declaration.
	\item \verb|<extraction>|\\
		A fact extraction.
	\item \verb|<condition>|\\
		A filtering condition.
	\item \verb|<duration>|\\
		The match duration.
	\item \verb|<consequence>|\\
		The rule match consequence.
	\item \verb|<semantics>|\\
		Load a semantic module.
\end{itemize}

\subsection{\texttt{drl:rules}}

The outtermost tag in each DRL file is the \verb|<rules>| tag.  It has
no attributes and serves only to aggregate \verb|<rule-set>|s and
\verb|<rule>|s. The XML namespace declaration for the base DRL should
be affixed to this element.

\begin{codelisting}
<drl:rules xmlns:drl="http://drools.org/rules">
\textcolor{light}{  <drl:rule-set ...>
    ...
  </drl:rule-set>
  <drl:rule ...>
    ...
  </drl:rule>}
</drl:rules>
\end{codelisting}

\subsection{\texttt{drl:rule-set}}

The \verb|<rule-set>| is a named container for \verb|<rules>|.  Its
only attribute is \verb|name| to provide for a name.  

\begin{codelisting}
<drl:rule-set name="Gold-Level Member Rules">
\textcolor{light}{  <drl:rule ...>
    ...
  </drl:rule>}
</drl:rule-set>
\end{codelisting}

\subsection{\texttt{drl:rule}}

The \verb|<rule>| tag is the most complex.  It \emph{must} contain
at least one \verb|<parameter>| and a \verb|<consequence>| tag.  It
may optionally contain \verb|<declaration>|, \verb|<extraction>|,
\verb|<condition>| and \verb|<duration>| tags.  The \verb|name|
attribute must be present. A \verb|<rule>| may exist inside either 
a \verb|<rules>| or \verb|<rule-set>| tag.

\begin{codelisting}
<drl:rule name="Over Credit Limit">
\textcolor{light}{  <drl:parameter ..>
    ...
  </drl:parameter>
  <drl:declaration ..>
    ...
  </drl:declaration>
  <drl:extraction ..>
    ...
  </drl:extraction>
  <drl:condition ..>
    ...
  </drl:condition>
  <drl:duration ..>
    ...
  </drl:duration>
  <drl:consequence ..>
    ...
  </drl:consequence>}
</drl:rule>
\end{codelisting}

\subsection{\texttt{drl:parameter}}

The \verb|<parameter>| tag defines a root fact object that the
rule expects to be provided from external resources.  The only
attribute is \verb|identifier| which provides the variable identifier
to be used to refer to the object elsewhere in the rule. The content
of the tag is dependent upon the semantic module used for the rule.
For illustration purposes, the Java Semantic Module has been used.

\begin{codelisting}
<parameter identifier="customer">
\textcolor{light}{  <java:class type="com.werken.Customer"/>}
</parameter>
<parameter identifier="account">
\textcolor{light}{  <java:class type="com.werken.Account"/>}
</parameter>
\end{codelisting}

\subsection{\texttt{drl:declaration}}

A \verb|<declaration>| tag is similar to a \verb|<parameter>| in
that it defines a typed and named object.  It must contain an
\verb|identifier| attribute to specify the name that may be used
to refer to the declared object elsewhere in the rule.  This tag
declares a variable that must be populated internally using an
\verb|<extraction>|.  For illustration purposes, the Java Semantic
Module has been used.

\begin{codelisting}
<drl:declaration identifier="custName">
\textcolor{light}{  <java:class type="java.lang.String"/>}
</drl:declaration>
<drl:declaration identifier="acctBalance">
\textcolor{light}{  <java:class type="java.math.BigInteger"/>}
</drl:declaration>
\end{codelisting}

\subsection{\texttt{drl:extraction}}

An \verb|<extraction>| defines a fact extraction.  Its only
attribute is \verb|target| which names the parameter or declaration
that the extracted fact should be assigned to.  The content
of the tag is dependent upon the semantic module used for the rule.
For illustration purposes, the Java Semantic Module has been used.

\begin{codelisting}
<drl:extraction target="customer">
\textcolor{light}{  <java:extractor>account.getCustomer()</java:extractor>}
</drl:extraction>
<drl:extraction target="custName">
\textcolor{light}{  <java:extractor>customer.getName()</java:extractor>} 
</drl:extraction>
<drl:extraction target="acctBalance">
\textcolor{light}{  <java:extractor>account.getBalance()</java:extractor>}
</drl:extraction>
\end{codelisting}

\subsection{\texttt{drl:condition}}

The \verb|<condition>| defines a condition that must be met in
order for the rule to match. It contains the main logic of the rule.
The content of the tag is dependent upon the semantic module used for the rule.
For illustration purposes, the Java Semantic Module has been used.

\begin{codelisting}
<drl:condition>
\textcolor{light}{  <java:condition>custName.equals( "McWhirter" )</java:condition>}
</drl:condition>
<drl:condition>
\textcolor{light}{  <java:condition>acctBalance.signum() == 0</java:condition>}
</drl:condition>>
\end{codelisting}

\subsection{\texttt{drl:duration}}

The \verb|<duration>| tag is used to specify a temporal condition.
The truth duration of a rule is the amount of time that all other
conditions must hold true before a match is determined.  The
content of the tag is dependent upon the semantic module used
for the rule. A simple \verb|<fixed-duration>| tag is supplied as
part of the base DRL syntax in order to specify static durations
that are not dependent upon rule data.

\begin{codelisting}
<drl:duration>
  <drl:fixed-duration seconds=".."
                      minutes=".."
                      hours=".."
                      days=".."
                      weeks=".."/>
</drl:duration>
\end{codelisting}

\subsection{\texttt{drl:consequence}}

The \verb|<consequence>| tag defines the action to be taken
once a rule matches for a set of root fact objects.  The content
of the tag is dependent upon the semantic module used for the rule.
For illustration purposes, the Java Semantic Module has been used.

\begin{codelisting}
<drl:consequence>
\textcolor{light}{  <java:consequence>
    account.addMoney( new BigInteger( "1000000" ) );
  </java:consequence>}
</drl:consequence>
\end{codelisting}


\subsection{\texttt{drl:semantics}}

The \verb|<semantics>| tag is used to load a semantic module.
It has no content and the \verb|module| attribute is required
in order to identify a semantic module to load.

\begin{codelisting}
<drl:semantics module="org.drools.semantics.java"/>
\end{codelisting}
