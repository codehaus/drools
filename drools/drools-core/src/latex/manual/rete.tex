\section{Rete}
\label{algo.rete}

Charles Forgy created the original Rete algorithm around 1981 as part
of his DARPA-funded research.  Compared to many previous
production-matching algorithms, Rete was very advanced.  Even today,
there have been few improvements to it in the general
case\footnote{Both ILOG and Haley claim to have optimized Rete
algorithms, but details are not currently public.}.  Variations on 
Rete, such as Treat, may have different performance characteristics
depending on the environment.  Some perform better with large rule 
sets but small numbers of objects, while other perform well for 
steady-state environments, but react poorly to numerous successive 
changes in the data.

A \emph{Rete network} is a graph through which data flows.
Originally, data was specified using Cambridge-prefix tuples since
Lisp-like languages were in style\footnote{As it is for many artificial
intelligence projects.}.  The tuples were used to express attributes
about objects.  For example, tuples may be used to express a person's
name and her pets.  The tuples are dropped into the Rete network,
and those that read the far end result in the firing of a rule.
The original production-matching was based upon pattern-matching and
could almost be accomplished completely through lexical means.

The Rete network is comprise of two types of nodes:

\begin{itemize}
	\item \emph{1-input/1-output nodes.}  The \emph{1/1} nodes are
		constrictive nodes that only allow matching tuples to
		flow through.  Any tuples that do not match are discarded
		by the node.
	\item \emph{2-input/1-output nodes.}
		The \emph{2/1} nodes simply connect the output arcs from two
		other nodes (either \emph{1/1} nodes or \emph{2/1} nodes) merging
		tuples from both the left and right incoming arcs
		into a single tuple on the outgoing arc.
\end{itemize}

There is a forest of \emph{1/1} nodes that act as the entry-point
into the entire Rete network for any incoming tuple.  The
network-entry  nodes filter tuples purely by their type.  
This allows for distinguishing tuples describing one type of 
entity from another.  Tuples about dogs and tuples about cats may 
each have a different type and will be differentiated from each 
other by the \emph{1/1} network-entry nodes.

Each condition of a rule is merely a pattern for a tuple type.
The condition describes the attributes that a tuple must have and acts
as a filter.  Each condition is transformed into a \emph{1/1} node
that only allows tuples matching the specified attributes.
An attribute value may be specified as a variable, which implies that
the variable must be the same value in all occurrences, for a 
combination of tuples.  The \emph{1/1} filter nodes are attached to
the network downstream from the \emph{1/1} that differentiates their
tuple type.

Consider a condition such as ``For any person who have a dog that
has the same name as that person's sister's cat, then...''  This could
be expressed with the condition patterns of:

\medskip

\begin{tabular}{llll}

(1) & \texttt{( person} & \texttt{name=sisOne?} & \texttt{sister=sisTwo? )}\\
(2) & \texttt{( person} & \texttt{name=sisTwo?} & \texttt{sister=sisOne? )}\\
(3) & \texttt{( person} & \texttt{name=sisOne?} & \texttt{cat=petName? )}\\
(4) & \texttt{( person} & \texttt{name=sisTwo?} & \texttt{dog=petName? )}\\

\end{tabular}

\medskip

Conditions \emph{\#1} and \emph{\#2} model the sister relationship so
that the rule only applies to two people who are sisters.  The
\verb|sisOne?| and \verb|sisTwo?| tokens are variables that must be
consistent across any set of tuples that match this rule.  

Conditions \emph{\#3} and \emph{\#4} serve two roles.  The \verb|cat|
and \verb|dog| attributes share the same \verb|petName?| variable and
serve to identify two people who have a cat and a dog with the same
name.  They each contain a \verb|name| attribute with either the
variable \verb|sisOne?| or \verb|sisTwo?| which ties the last two
conditions back to the first two.

\begin{figure}[htbpc]
  \begin{center}
	\xymatrix {
		\bullet \ar[d] \\
		type(person) \ar[d] \ar[dr] \ar[drr] \ar[drrr] \\
%
	condition(1) \ar[dr] & condition(2) \ar[d] & condition(3) \ar[dd] & condition(4) \ar[ddd] \\
%
	       & join(1) \ar[dr] \\
%
	       & & join(2) \ar[dr] \\
%
	       & & & join(3) \ar[d] \\
%
	       & & & terminal \\
	}
  \end{center}
\end{figure}

\setlength{\extrarowheight}{3pt}


\begin{figure}[htbpc]
  \begin{center}
    \begin{tabular}{|c||c|c|c|c|}
      \hline
        \textsf{\emph{type}} & \textsf{name} & \textsf{sister} & \textsf{cat} & \textsf{dog} \\
      \hline
      \hline
        \multicolumn{5}{|c|}{\emph{tuple set \# 1}}\\
      \hline 
      \hline 
        person & rebecca & jeannie & ugly & \emph{null} \\
      \hline
        person & jeannie & rebecca & \emph{null} & ugly \\
      \hline
      \hline
        \multicolumn{5}{|c|}{\emph{tuple set \# 2}}\\
      \hline
      \hline
        person & rebecca & jeannie & zoomie & \emph{null} \\
      \hline
        person & jeannie & rebecca & \emph{null} & ugly \\
      \hline
    \end{tabular}
  \end{center}

  \caption{Example tuple sets}
  \label{table.tuplesets}
\end{figure}

If two sets of tuples (see Figure \ref{table.tuplesets}) were
asserted against the rule, \emph{tuple set \#1} would cause a firing
of the rule, where \emph{tuple set \#2} would not.  In both cases,
the two tuples would pass nodes \emph{(1)} and \emph{(2)}, as the
nodes simply associate the \verb|sisOne?| and \verb|sisTwo?| variables
with the appropriate values from each tuple.

The \emph{join (1)} node would allow both tuples to merge and
propagate past it in both the first and second case.  Additionally,
for both cases, the \emph{rebecca} tuple would pass node \emph{(3)}
and the \emph{jeannie} tuple would pass node \emph{4}.

The \emph{join (3)} node is where the two cases differ.  In the first
case, nodes \emph{(3)} and \emph{(4)} have each associated the value
of ``ugly'' to the \verb|petName?| variable.  In the second case, the
two nodes has assigned different values to the variable.  The
\emph{join (3)} node only allows those tuples that have consistent
associations with all variables to pass.
