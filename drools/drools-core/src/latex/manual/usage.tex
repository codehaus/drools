\chapter{Usage}

\section{\drools{} Client API}

\subsection{Introduction}

\drools{} is divided into several sets of APIs.  The core client API
is by far the simplist and most commonly used by developers.  The
core \drools{} client API consists of locating a rule-base, creating
a working memory, and then managing fact assertion, modifcation and
retraction.

\subsection{Locating a Rule-base}

\verb|RuleBase| objects are typically loaded from a
\verb|RuleBaseRepository|.  Different repository implementations
provide different mechanisms for the storage of each
\verb|RuleBase|.  The method by which your project obtains
a \verb|RuleBaseRepository| is implementation-specific but
may involve a lookup and discovery mechanism such as JNDI.

Once a \verb|RuleBaseRepository| has been obtained, the
simple method\\ \verb|lookupRuleBase(..)| method is used
to retrieve a \verb|RuleBase| by its URI.\footnote{URIs that
identify rule bases are not necessarily deferenceable.  They
serve only as unique identifiers for a collection of rules.}

\bigskip
\begin{codelisting}
RuleBaseRepository repo = myUtilities.getRepository();

String ruleBaseUri = "http://rules.werken.com/family-relationships";

RuleBase ruleBase = repo.lookupRuleBase( ruleBaseUri );
\end{codelisting}

\newpage

\subsection{Creating a Working Memory}

\drools{} provides two different types of working memory
implementations: a normal \verb|WorkingMemory| and a
\verb|TransactionWorkingMemory|.

\begin{itemize}

	\item \textbf{\textsf{WorkingMemory}} \\
		The normal \verb|WorkingMemory| implementation 
		propagates fact assertions, modifications and
		retractions through the Rete-OO network in
		real-time.  Once the fact manipulation 
		methods return control to the client program,
		all facts have been assimilated and acted upon.

	\item \textbf{\textsf{TransactionalWorkingMemory}} \\
		The \verb|TransactionalWorkingMemory| does \emph{not}
		propagate fact manipulation information through the
		Rete-OO in real-time.  Instead, it calculate the net
		fact changes and performs all manipulations immediately
		upon usage of the \verb|commit()| method.  No actions
		are performed until \verb|commit()| is called, and
		all fact information is discarded if \verb|abort()|
		is used.

\end{itemize}

The methods on \verb|RuleBase| to construct working memories are:

\bigskip

\begin{codelisting}    
/** Create a WorkingMemory session for this RuleBase.
 *
 *  @see WorkingMemory
 *
 *  @return A newly initialized WorkingMemory.
 */
public WorkingMemory createWorkingMemory() 

/** Create a TransactionalWorkingMemory session for this RuleBase.
 *
 *  @see TransactionalWorkingMemory
 *
 *  @return A newly initialized TransactionalWorkingMemory.
 */
public TransactionalWorkingMemory createTransactionalWorkingMemory()
\end{codelisting}

\subsection{Fact Manipulation}

Once you have a \verb|WorkingMemory| in hand, you must assert fact
objects to make them available to \drools{} for analysis.  Additionally, as facts
change, the engine must be notified.  Likewise, when a object should
no longer be considered for analysis, it must be retracted from
the engine.  Method for these three actions are defined upon the
\verb|WorkingMemory| class.

\bigskip

\footnotesize
\begin{alltt}
/** Assert a new fact object into this working memory.
*
*  @param object The object to assert.
*
*  @throws AssertionException if an error occurs during assertion.
*/
public void assertObject(Object object) throws AssertionException
\newpage
/** Modify a fact object in this working memory.
*
*  With the exception of time-based nodes, modification of
*  a fact object is semantically equivelent to retracting and
*  re-asserting it.
*
*  @param object The object to modify.
*
*  @throws FactException if an error occurs during modification.
*/
public void modifyObject(Object object) throws FactException

/** Retract a fact object from this working memory.
*
*  @param object The object to retract.
*
*  @throws RetractionException if an error occurs during retraction.
*/
public void retractObject(Object object) throws RetractionException
\end{alltt}
\normalsize

\section{JSR-94 API}

The \drools{} project is currently working on a JSR-94 API binding.

