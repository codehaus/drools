\section{Rules}

Many enterprise software systems today already include the concept of
rules.  Most times, these rules are directly implemented in code and
are difficult to adapt to a changing business landscape.  The domain
model many times includes business logic which may change often. When
these business 'rules' are coded using normal systems programming
techniques\footnote{Many times, a long sequence of
\texttt{if}/\texttt{else} statements is used to realize business
rules.}, modification and maintenance of the logic can become
difficult.

Examples of typical simple business rules include:

\begin{itemize}

	\item When a customer applies for a loan of more than \$80,000 
		and has less than \$5,000 in their savings account and
		has had the account for less than 3 years, then reject
		the loan application.

	\item When a customer orders a box of goods and a routed delivery 
		vehicle can include the delivery in today's route by
		lengthening the route by no more than 8 miles, then add
		the customer's delivery to the vehicle.

	\item When a trouble-ticket from a high-priority customer has
		been unresolved for 60 minutes, escalate the urgency of the
		ticket and notify the shift manager.

	\item When a customer buys pie, suggest that they might enjoy
		some ice-cream.
		
\end{itemize}

More complex rules that involve multiple participants or chunks of data
may also exist in systems:

\begin{itemize}

	\item When someone is selling a product desired by another person
		for a price less-than-or-equal-to the price the other is
		is willing to pay, notify both parties.

	\item When an author submits an article and either three junior
		editors a single senior editor has signed off on it, send
		the article to the production department work queue.

	\item When email not directly addressed to me arrives and the
		sender is not on my 'approved' list, then direct it to
		my mail folder that holds potential spam.

\end{itemize}

As companies are often changing the way they do business,  responding
quickly is important.  Changing business logic that
is realized in compiled code can become quite an arduous job.  Systems
that respond to changes in policy and promotions as quickly as 
the enterprise makes decisions reduce maintenance costs and 
development cycle times.

