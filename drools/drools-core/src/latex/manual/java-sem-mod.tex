\chapter{Java Semantic Module}

\section{Overview}

The Java Semantic Module provides implementations of semantic
components that adhere to the Java language semantics.  The components
can be used directly from Java through a DRL file.


\begin{itemize}
	\item \textbf{\textsf{org.drools.semantics.java.ClassObjectType}}\\
		A \verb|ObjectType| implementation that adheres
		to Java class types.  Usable within \verb|<parameter>|
		and \verb|<declaration>| DRL tags.
	\item \textbf{\textsf{org.drools.semantics.java.ExprCondition}}\\
		A \verb|Condition| implementation that uses boolean
		Java expressions for filtering. Usable within 
		\verb|<condition>| DRL tags.
	\item \textbf{\textsf{org.drools.semantics.java.ExprExtractor}}\\
		A \verb|Extractor| implementation that uses Java
		expressions for extracting new facts.  Usable
		within \verb|<extraction>| DRL tags.
	\item \textbf{\textsf{org.drools.semantics.java.BlockConsequence}}\\
		A \verb|Consequence| impementation that uses
		A block of Java statements as the action of a
		matched rule.  Usable within \verb|<consequence>|
		DRL tags.
\end{itemize}

\clearpage

\section{Usage with DRL}

\subsection{Loading the Module}

The Java Semantic Module's tags exist within the XML namespace
URI of \verb|http://drools.org/semantics/java| and within the
Java package of\\ \verb|org.drools.semantics.java|.  To use the
Java Semantic Module within a DRL file, the DRL \verb|<semantics>|
tag must be used, as must an XML namespace prefix binding.

\begin{codelisting}
\textcolor{light}{<drl:rules xmlns:drl="http://drools.org/rules"}
       xmlns:java="http://drools.org/semantics/java"\textcolor{light}{>}

  <drl:semantics module="org.drools.semantics.java"/>

  \textcolor{light}{<drl:rule ...>
    <drl:parameter identifier="account">}
      <java:class type="com.werken.Account"/>
    \textcolor{light}{</drl:parameter>
  </drl:rule>

</drl:rules>}
\end{codelisting}


\subsection{\texttt{java:class}}

The \verb|<java:class>| tag defines an \emph{object type} that adheres
to Java class semantics for types.  It has a single attribute of
\verb|type| which takes a class name as a value.  The tag may be
used as the content of both \verb|<drl:parameter>| and
\verb|<drl:declaration>| DRL tags.

\begin{codelisting}
\textcolor{light}{<drl:parameter identifier="account">}
  <java:class type="com.werken.Account"/>
\textcolor{light}{</drl:parameter>
<drl:declaration identifier="person">}
  <java:class type="com.werken.Person"/>
\textcolor{light}{</drl:declaration>}
\end{codelisting}

\subsection{\texttt{java:condition}}

The \verb|<java:condition>| tag defines an \emph{condition} that
adheres to Java boolean expression semantics.  It has no
attributes and the body content is the boolean expression to
evaluate.  The tag may be used as the content of a \verb|<drl:condition>|
tag.

\begin{codelisting}
\textcolor{light}{<drl:condition>}
  <java:condition>acctBalance == 0</java:condition>
\textcolor{light}{</drl:condition>}
\end{codelisting}

\subsection{\texttt{java:extractor}}

The \verb|java:extractor| tags defines a \emph{fact extractor}
that adheres to Java expression semantics.  It has
no attributes and the body content is the expression to generate
the new fact.  The tag may be used as the content of a 
\verb|<drl:extraction>| tag.

\begin{codelisting}
\textcolor{light}{<drl:extraction target="accountBalance">}
  <java:extractor>person.getAccount().getBalance()</java:extractor>
\textcolor{light}{</drl:extraction>}
\end{codelisting}

\subsection{\texttt{java:consequence}}

The \verb|<java:consequence>| tag defines a \emph{rule consequence}
that adheres to Java statement block semantics.  It may exist as the
content of a \verb|<drl:consequence>| tag.  It has no
attributes and the body content is the set of statements
to execute upon rule match.


\begin{codelisting}
\textcolor{light}{<drl:consequence>}
   System.err.println( "The balance is: " + acctBalance );
   theRepoMan.addAccount( account );
   assertObject( theRepoMan );
\textcolor{light}{</drl:consequence>}
\end{codelisting}

