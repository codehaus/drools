\chapter{Administrative API}

The administrative functions of drools include the assembly
and deployment of rules.  These activities may certainly be
performed by the same application that acts as the client,
The API has been designed to allow for the separation between
the client and administrative responsibilities.

\section{Building rules}

Programmatically building rules using the Java API is covered
in Chapter~\vref{rule.assembly}.  The next section discusses
loading rules that were built using the Drools Rule Language,
which is described in depth in Chapter~\vref{drl}.

\section{Loading rules}
\label{admin.rules.loading}

Rule-sets can be loaded using the \indexClass{RuleSetReader} from
a variety of sources including \class{URL}, \class{InputStream} and
\class{Reader} objects.  A \class{RuleSetReader} builds a
\indexClass{RuleSet} from the contents of a DRL document (Chapter~\vref{drl}).

The \indexClass{RuleSetReader} by default may use any and all semantic
modules available on the classpath (see Chapter~\vref{smf}).  It may
be optionally parameterized with a \indexClass{SAXReader} and a
\indexClass{SemanticsRepository}.  By default, the
\indexClass{RuleSetReader} uses the Java API for XML Parsing (JAXP) \index{JAXP}
for reading the XML documents. See Figure~\vref{rulesetreader} for an
example of loading a \indexClass{RuleSet} from a \class{URL}.

\begin{figure}
\begin{javaCodelisting}
RuleSetReader reader     = new RuleSetReader();
URL           ruleSetUrl = new URL( "http://myco.com/theRuleSet.xml" );
RuleSet       ruleSet    = reader.read( ruleSetUrl );
\end{javaCodelisting}
\caption{Loading a \indexClass{RuleSet} from a \class{URL}}
\label{rulesetreader}
\end{figure}

\section{Building a \indexClass{RuleBase}}

\section{Deploying a \indexClass{RuleBase} to a JNDI\index{JNDI} data store}

\section{Serializing a \indexClass{RuleBase} to a file}

