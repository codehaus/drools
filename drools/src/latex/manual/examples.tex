\chapter{Sisters and Pets}

\lstset{emph={RuleBase,RuleBaseBuilder,WorkingMemory},emphstyle=\color{green}}
\lstset{basicstyle=\tiny,numbers=left,frame=single,fancyvrb=true,keywordstyle=\color{blue}}
\chapter{Fibonacci Calculation}
\section {Background}
See \href{http://www.mcs.surrey.ac.uk/Personal/R.Knott/Fibonacci/fibnat.html}{Fibonacci Numbers and Golden section in Nature} for more information.  
\\
\begin {tabular}{|c|}
\hline
The Fibonacci series is formed by adding the latest two numbers\\
to get the next one, starting from 0 and 1:\\
\\
0, 1, 1, 2, 3, 5, 8, 13, 21, 34, 55, 89, 144, 233, 377, 610, 987, ...\\ 
\hline
\end {tabular}

\section {Making Drools work}
\subsection {Loading Drools rule(s)}
Notice when specifying a rules location, it doesn't have to be local.  
\lstLine {../../../drools-examples/src/main/org/drools/examples/fibonacci/FibonacciExample.java} {FibonacciExample.java} {firstline=14,lastline=15}

\subsection {How does Drools find my objects}
Drools isn't magic.
It needs to know where to find the objects to inspect. 
This also gives the programmer more flexibility, you can pick and choose what the rules engine should look at. 
\lstLine {../../../drools-examples/src/main/org/drools/examples/fibonacci/FibonacciExample.java} {FibonacciExample.java} {firstline=16,lastline=23}

\subsection {Telling Drools to start}
Drools is now running. 
Notice, that the next lines which are run are not lines 26 or 27. 
Drools will start calling your event handlers, i.e. rules.
All of your rules will run, then the program will continue on to the next lines.
\lstLine {../../../drools-examples/src/main/org/drools/examples/fibonacci/FibonacciExample.java} {FibonacciExample.java} {firstline=24,lastline=27}



\clearpage
\section {Sample Code}
\subsection {Fibonacci}
\lst {../../../drools-examples/src/main/org/drools/examples/fibonacci/Fibonacci.java} {Fibonacci.java}

\clearpage
\subsection {Fibonacci Example}
\lst {../../../drools-examples/src/main/org/drools/examples/fibonacci/FibonacciExample.java} {FibonacciExample.java}

\clearpage
\subsection {Fibonacci DRL}
\lst {../../../drools-examples/src/main/org/drools/examples/fibonacci/fibonacci.drl} {Fibonacci.drl}





\cleardoublepage
\chapter{Trouble-Ticket Escalation}

\cleardoublepage
\chapter{Manners Benchmark}



\cleardoublepage
\chapter{Fish Monger}
%\lstset{basicstyle=\tiny,numbers=left,frame=single,fancyvrb=true,keywordstyle=\color{blue}}
\section{Java Code}
%%\lstLine {../java/examples/fishmonger/FishMonger.java} {FishMonger code} {firstline=18,lastline=24}
\subsection{Creating a Rule Base}
%%\lstLine {../java/examples/fishmonger/FishMonger.java} {FishMonger code} {firstline=25,lastline=30}

\subsection{Loading the Rules---XML file}
%%\lstLine {../java/examples/fishmonger/FishMonger.java} {FishMonger code} {firstline=31,lastline=41}

\subsection{Adding the Rules to the Rules base}
%%\lstLine {../java/examples/fishmonger/FishMonger.java} {FishMonger code} {firstline=42,lastline=48}

\subsection{Feature Request---Two step Rules base}
I hope all the above can become something like\\
final URL url = FishMonger.class.getResource("fishmonger.drl");\\
final RuleSet rules = RulesSetLoader.getRuleSet(url);


\clearpage
\section{Working Memory 101}
%%\lstLine {../java/examples/fishmonger/FishMonger.java} {FishMonger code} {firstline=49,lastline=53}

\subsection{Telling the Engine about your Objects}
%%\lstLine {../java/examples/fishmonger/FishMonger.java} {FishMonger code} {firstline=54,lastline=91}

\subsection{Telling the Engine to Run!}
%%\lstLine {../java/examples/fishmonger/FishMonger.java} {FishMonger code} {firstline=99,lastline=105}




\clearpage
%\lstset{emph={parameter,consequence,condition,rule,rules},emphstyle=\color{green}}
\section{Drools Rule Logic (.drl) File}
\subsection{Explode Cart}
%%\lstLine {../java/examples/fishmonger/fishmonger.drl} {FishMonger DRL} {firstline=9,lastline=33}
\subsection{Free Fish Food Special}
%%\lstLine {../java/examples/fishmonger/fishmonger.drl} {FishMonger DRL} {firstline=34,lastline=61}

\clearpage
\subsection{Suggest a Tank}
%%\lstLine {../java/examples/fishmonger/fishmonger.drl} {FishMonger DRL} {firstline=62,lastline=83}
\subsection{Apply a Discount}
%%\lstLine {../java/examples/fishmonger/fishmonger.drl} {FishMonger DRL} {firstline=84,lastline=101}
\clearpage
\subsection{Source code of Fish Monger.drl}
%\lst {../java/examples/fishmonger/fishmonger.drl} {FishMonger DRL}






\cleardoublepage
\section {Output}
\begin{lstlisting}{fishmonger}

--------------------------------------------------
 R U N N I N G  fishmonger.FishMonger
--------------------------------------------------

loading: jar:file[this is specific to your computer!!!]
----------------------------------------
    PRE
----------------------------------------
[ShoppingCart:
	      gross total=90.92999999999999
	 discounted total=90.92999999999999
	[CartItem: name='tropical fish'; cost=12.99]
	[CartItem: name='tropical fish'; cost=12.99]
	[CartItem: name='tropical fish'; cost=12.99]
	[CartItem: name='tropical fish'; cost=12.99]
	[CartItem: name='tropical fish'; cost=12.99]
	[CartItem: name='tropical fish'; cost=12.99]
	[CartItem: name='tropical fish'; cost=12.99]
]
----------------------------------------
Examining each item in the shopping cart.
Adding free tropical fish food sample to cart
Adding free tropical fish food sample to cart
Adding free tropical fish food sample to cart
Adding free tropical fish food sample to cart
Adding free tropical fish food sample to cart
Adding free tropical fish food sample to cart
Adding free tropical fish food sample to cart
*** SUGGESTION: Would you like to buy a tank for your7 fish?
Applying 15% discount to cart
----------------------------------------
    POST
----------------------------------------
[ShoppingCart:
	      gross total=90.92999999999999
	 discounted total=77.2905
	[CartItem: name='tropical fish'; cost=12.99]
	[CartItem: name='tropical fish'; cost=12.99]
	[CartItem: name='tropical fish'; cost=12.99]
	[CartItem: name='tropical fish'; cost=12.99]
	[CartItem: name='tropical fish'; cost=12.99]
	[CartItem: name='tropical fish'; cost=12.99]
	[CartItem: name='tropical fish'; cost=12.99]
	[CartItem: name='tropical fish food sample'; cost=0.0]
	[CartItem: name='tropical fish food sample'; cost=0.0]
	[CartItem: name='tropical fish food sample'; cost=0.0]
	[CartItem: name='tropical fish food sample'; cost=0.0]
	[CartItem: name='tropical fish food sample'; cost=0.0]
	[CartItem: name='tropical fish food sample'; cost=0.0]
	[CartItem: name='tropical fish food sample'; cost=0.0]
]
----------------------------------------
\end{lstlisting}

\cleardoublepage
\section {Sample Code}
\subsection {Cart Item}
%\lst {../java/examples/fishmonger/CartItem.java} {Cart Item code}

\cleardoublepage
\subsection {FishMonger}
%\lst {../java/examples/fishmonger/FishMonger.java} {FishMonger code}

\cleardoublepage
\subsection {Shopping Cart}
%\lst {../java/examples/fishmonger/ShoppingCart.java} {Shopping Cart code}























\chapter{Family Tree of Baggins}
\definecolor {CarnationPink} {cmyk}{0,0.63,0,0}
\begin{figure}[!hbp]
\center
	\begin {displaymath}
	\xymatrix {
	&& Balbo\ar[dll] \ar[dr]\\
Mungo \ar [d] &	\textcolor{CarnationPink} {Laura}	&& Lar \ar[d] & \textcolor{CarnationPink} {Tanta}\\
Bungo \ar[d] &  \textcolor{CarnationPink} {Belladonna}	&& Fosco \ar[d]& \textcolor{CarnationPink} {Ruby}\\
Bilbo					&&& Drogo\ar[d] & \textcolor{CarnationPink} {Primula} \\
							&&& Frodo
}	\end {displaymath}
\caption{Baggins Family Tree}
\end{figure}
\par
See \href {http://www.tolkienonline.de/etep/Tolkien/FamilyTrees/Bilbo.html} {http://www.tolkienonline.de/etep/Tolkien/FamilyTrees/Bilbo.html}

and \href {http://www.tuckborough.net/baggins.html}
{http://www.tuckborough.net/baggins.html} for more info

Please note that \textcolor{CarnationPink} {Belladonna} is female.  This will be important later!



%\lstset{basicstyle=\tiny,numbers=left,frame=single,fancyvrb=true,keywordstyle=\color{blue}}
\section{Java Code}



\cleardoublepage
\section {Sample Code}
\subsection {Person}
%\lst {../java/examples/familyTree/Person.java} {Person code}
\cleardoublepage
\subsection {Main}
%\lst {../java/examples/familyTree/Main.java} {Main code}




\cleardoublepage
\subsection{Source code of Family Tree.drl}
%\lst {../java/examples/familyTree/familyTree.drl} {FamilyTree DRL}


