\chapter{Project Information}


\section{Web Site}

All development resources related to Drools are hosted by 
\textbf{The Codehaus}\index{Codehaus}, the open-source 
arm of The Werken Company. Drools maintains a website at:

\begin{quote}
	\url{http://drools.org/}
\end{quote}


\section{Mailing Lists}

The drools project maintains two mailing lists.  The first, known as
\verb|drools-interest| is for general discussion by users and
developers of drools.  The second list is \verb|drools-cvs| which
simply tracks changes made to the source-code through the CVS
repository. For information about subscribing to each list or access 
to the list archives:

\begin{quote}
    \url{http://lists.codehaus.org/listinfo/drools-interest}\\
    \url{http://lists.codehaus.org/listinfo/drools-cvs}
\end{quote}

\section{Source Repository}

The drools project maintains a revision control repository using
CVS.  To checkout the latest sources, you must issue two CVS commands.
The first is used to login.  When presented with a prompt for a
password, simply press \emph{ENTER}.

{\small
\begin{verbatim}
  cvs -d:pserver:anonymous@cvs.codehaus.org:/scm/cvspublic login
  cvs -d:pserver:anonymous@cvs.codehaus.org:/scm/cvspublic co drools
\end{verbatim}
}

\clearpage

\section{Internet Relay Chat}

There is a dedicated channel on The Werken Company's IRC server for
drools:\\

\begin{tabular}{rl}
address & \verb|irc.codehaus.org| \\
port    & \verb|6667| \\
channel & \verb|#drools|\\
url     & \url{irc://irc.codehaus.org:6667/drools}\\
\end{tabular}

\bigskip

\section{Bug, Issue \& Feature Tracking}

For bug, issue and feature tracking, the Drools project uses the
Jira project management system provided by The Codehaus.

\begin{quote}
    \url{http://jira.codehaus.org/}
\end{quote}

\clearpage

\section{Project Team}

\subsection{Bob McWhirter}

Bob McWhirter originally founded the Drools project in 2000 and
developed the Rete-OO algorithm used by the engine.  Bob is also
the founder of The Werken Company and the chief architect behind
the commercial \textbf{Fluxtapose} suite of tools which build
upon Drools to provide a complete solution for implementing
business rules.

\subsection{Thomas Diesler}

Thomas Diesler researched and supplied the JSR-94 Rule-Engine API
bindings for Drools.

\subsection{Roger F. Gay}

Roger F. Gay devised the XML Schemas for the core DRL syntax and
each semantic module.

\subsection{Contributors}

Others have contributed ideas, patches and testing assistance
over the years:

\begin{itemize}
  \item Dave Cramer \emph{(eBox)}
  \item Martin Hald
  \item Matt Ho
  \item Pete Kazmier \emph{(iBasis)}
  \item Christiaan ten Klooster
  \item James Roome
  \item Bart Selders \emph{(iBanx)}
  \item James Strachan \emph{(CoreDevelopers Network)}
  \item Tom Vasak
\end{itemize}

