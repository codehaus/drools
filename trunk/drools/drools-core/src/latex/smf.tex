\chapter{Semantic Module Framework}
\label{smf}

The Semantic Module Framework\index{semantic module framework} (SMF)
provides an extension point to the Drools Rule Language syntax
(see Chapter \vref{drl}).  By creating semantic modules,
domain-specific rule types, object-types, conditions, extractors and consequences
can be added to the core DRL language.

\section{Semantic components}

\subsection{\indexClass{RuleType}}

\subsection{\indexClass{ObjectType}}

Implementations of \indexClass{ObjectType} allow rules to be
defined in terms of semantic types. Instead of requiring all
rules to be defined in terms of Java classes, rules may be
defined in terms of higher semantics.  For example, while
all XML documents may be instances of \texttt{org.w3c.dom.Document}
each document may have a different semantic type based upon
the name and namespace of the root tag.  Given an \texttt{Object},
an \indexClass{ObjectType} implementation must simply determine
if it matches its semantic type (Figure \vref{objecttype}).

\begin{figure}
\begin{javaCodelisting}
package org.drools.spi;

public interface ObjectType
    extends SemanticComponent
\{
    boolean matches(Object object);
\}
\end{javaCodelisting}
\caption{\indexClass{ObjectType} interface}
\label{objecttype}
\end{figure}

\subsubsection{\indexClass{ConfigurableObjectType}}

Configuration information may be passed to an \indexClass{ObjectType}
if the implementation is marked with the
\indexClass{ConfigurableObjectType} interface.
A \indexMethod{ConfigurableObjectType}{configure(...)} method
is added by the \indexClass{ConfigurableObjectType} interface.
through which additional information may be passed
(Figure~\vref{configurableobjecttype}).

\begin{figure}
\begin{javaCodelisting}
package org.drools.smf;

public interface ConfigurableObjectType
    extends ObjectType
\{
    void configure(Configuration config) throws ConfigurationException;
\}
\end{javaCodelisting}
\caption{\indexClass{ConfigurableObjectType} interface}
\label{configurableobjecttype}
\end{figure}



\subsection{\indexClass{Condition}}

\subsubsection{\indexClass{ConfigurableCondition}}

\subsection{\indexClass{Extractor}}

\subsubsection{\indexClass{ConfigurableExtractor}}

\subsection{\indexClass{Condition}}

\subsubsection{\indexClass{ConfigurableCondition}}

\subsection{\indexClass{Consequence}}

\subsubsection{\indexClass{ConfigurableConsequence}}

\section{The \indexClass{Configuration} structure}

For the \texttt{Configurable...} form of the semantic components, 
configuration information is communicated through a tree of
\indexClass{Configuration} objects.  Each \class{Configuration}
object acts as a node in the tree, and may contain the following
data:

\begin{itemize}
  \item \textbf{Attributes.} Zero or more name/value pairs of strings
(Figure~\vref{configuration.attributes}).
  \item \textbf{Text.} A single string text value
(Figure~\vref{configuration.text}).
  \item \textbf{Child \class{Configuration} nodes.} Zero or more named
    child \class{Configuration} nodes
(Figure~\vref{configuration.children}).
\end{itemize}

\begin{figure}
\begin{javaCodelisting}
String[] attrNames = config.getAttributeNames();
String   someValue = config.getAttribute( someName );
\end{javaCodelisting}
\caption{Attribute-related operations of \indexClass{Configuration}}
\label{configuration.attributes}
\end{figure}

\begin{figure}
\begin{javaCodelisting}
String nodeText = config.getText();
\end{javaCodelisting}
\caption{Text-related operation of \indexClass{Configuration}}
\label{configuration.text}
\end{figure}

\begin{figure}
\begin{javaCodelisting}
Configuration[] allChildren    = config.getChildren();
Configuration   someFirstChild = config.getChild( someName );
Configuraiton[] someChildren   = config.getChildren( someName );
\end{javaCodelisting}
\caption{Child-related operations of \indexClass{Configuration}}
\label{configuration.children}
\end{figure}

The tree of \indexClass{Configuration} nodes may be thought of as
a simplified version of an XML structure.  For configurable semantic
components used through the DRL (Chapter~\vref{drl}), the root 
\class{Configuration} is based upon the component's own tag, and
children tags are represetned by children \class{Configuration}
nodes.

\section{Semanitc Module Descriptor}
\label{module.descriptor}


\section{\indexClass{SemanticsRepository}}

A \indexClass{SemanticsRepository} manages a set of
\indexClass{SemanticModule} objects and allows each to be looked-up
by its URI (Figure~\vref{semanticsrepo}).  Primary a \indexClass{SemanticsRepository} is used by
a \indexClass{RuleSetReader} (Section~\vref{admin.rules.loading})
in order to extend the core DRL (Chapter~\vref{drl}) syntax.

\begin{figure}
\begin{javaCodelisting}
SemanticsRepository repo       = locateSemanticsRepository();
SemanticModule[]    modules    = repo.getSemanticModules();
SemanticModule      someModule = repo.lookupSemanticModule( someUri );
\end{javaCodelisting}
\caption{Usage of the \indexClass{SemanticsRepository}}
\label{semanticsrepo}
\end{figure}

\section{The \indexClass{DefaultSemanticsRepository} helper}

The \indexClass{DefaultSemanticsRepository} helper class is useful
in that it contains all conforming semantic modules available on
the classpath.  Each module that has a module descriptor
(Section~\vref{module.descriptor}) located as
\texttt{META-INF/drools-semantics.xml}
will be automatically discovered by the
\class{DefaultSemanticsRepository}
upon first use.  

Being a help class that is initialized once, it follows the
singleton pattern.  To use the
\indexClass{DefaultSemanticsRepository}, the
\indexMethod{DefaultSemanticsRepository}{getInstance()}
method will retrieve the singleton instance
(Figure~\vref{defaultsemanticsrepo}).

\begin{figure}
\begin{javaCodelisting}
SemanticsRepository repo    = DefaultSemanticsRepository.getInstance();
SemanticModule[]    modules = repo.getSemanticModules();
\end{javaCodelisting}
\caption{Retrieving and using the
\indexClass{DefaultSemanticsRepository} helper}
\label{defaultsemanticsrepo}
\end{figure}

