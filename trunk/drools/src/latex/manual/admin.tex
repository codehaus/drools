\chapter{Administrative API}
\label{admin.api}

\section{Introduction}

The administrative functions of drools include the assembly
and deployment of rules.  These activities may certainly be
performed by the same application that acts as the client,
The API has been designed to allow for the separation between
the client and administrative responsibilities.

\section{Building rules}

Programmatically building rules using the Java API is covered
in Chapter~\vref{rule.assembly}.  The next section discusses
loading rules that were built using the Drools Rule Language,
which is described in depth in Chapter~\vref{drl}.

\section{Building a \indexClass{RuleBase}}

\subsection{Loading rules}
\label{admin.rules.loading}

Rule-sets can be loaded using the \indexClass{RuleSetReader} from
a variety of sources including \class{URL}, \class{InputStream} and
\class{Reader} objects.  A \class{RuleSetReader} builds a
\indexClass{RuleSet} from the contents of a DRL document (Chapter~\vref{drl}).

The \indexClass{RuleSetReader} by default may use any and all semantic
modules available on the classpath (see Chapter~\vref{smf}).  It may
be optionally parameterized with a \indexClass{SAXReader} and a
\indexClass{SemanticsRepository}.  By default, the
\indexClass{RuleSetReader} uses the Java API for XML Parsing (JAXP) \index{JAXP}
for reading the XML documents. See Figure~\vref{rulesetreader} for an
example of loading a \indexClass{RuleSet} from a \class{URL}.

\begin{figure}
\begin{javaCodelisting}
RuleSetReader reader     = new RuleSetReader();
URL           ruleSetUrl = new URL( "http://myco.com/theRuleSet.xml" );
RuleSet       ruleSet    = reader.read( ruleSetUrl );
\end{javaCodelisting}
\caption{Loading a \indexClass{RuleSet} from a \class{URL}}
\label{rulesetreader}
\end{figure}

\subsection{Conflict resolution}

When constructing a \indexClass{RuleBase}, a \emph{conflict resolution
strategy} must be selected.  Conflict resolution is described fully in
Chapter~\vref{conflict}, but a summary of available strategies appears
in Figure~\vref{admin.conflict.strategies}.

\begin{figure}[hb]
\begin{center}
\begin{tabular}{|l|p{0.4\textwidth}|}
\hline
\textsf{\textbf{Class}} & \textsf{\textbf{Description}} \\
\hline
\footnotesize
\indexClass{SalienceConflictResolutionStrategy} & 
  \footnotesize Resolve conflicts based upon the 
  \emph{salience} of the rules. If two conflicting rules 
  have the same salience, one is selected 
  \emph{at random} to be fired first.\\
\hline
\footnotesize
\indexClass{ComplexityConflictResolutionStrategy} & 
  \footnotesize
  Resolves conflicts based upon the complexity of 
  the rules as measured by the number of conditions.  
  Rules with more conditions have higher priority than 
  those with fewer conditions.  If two conflicting rules 
  have the same measure of complexity, conflict is then 
  resolved as with the salience conflict resolution strategy. \\
\hline
\footnotesize
\indexClass{SimplicityConflictResolutionStrategy} & 
  \footnotesize
  Resolves conflicts based upon the simplicity of 
  the rules as measured by the number of conditions.  
  Rules with fewer conditions have higher priority than 
  those with more conditions.  If two conflicting rules 
  have the same measure of complexity, conflict is then 
  resolved as with the salience conflict resolution strategy. \\
\hline
\end{tabular}
\end{center}
\caption{Conflict-resolution strategies}
\label{admin.conflict.strategies}
\end{figure}

\begin{figure}
\begin{javaCodelisting}
RuleBaseBuilder builder = new RuleBaseBuilder();

builder.addRuleSet( theRuleSet );

builder.setConflictResolutionStrategy( 
                            SalienceConflictResolutionStrategy.getInstance() );

RuleBase ruleBase = builder.buildRuleBase();
\end{javaCodelisting}
\caption{Using \indexClass{RuleBaseBuilder} to build a \indexClass{RuleBase} from a \indexClass{RuleSet}}
\label{rulebasebuilder}
\end{figure}

\section{Deploying a \indexClass{RuleBase} to a JNDI\index{JNDI} data store}

A fully-constructed \indexClass{RuleBase} may be serialized and stored
within a JNDI-accessible \index{JNDI} directory
(Figure~\vref{deploy.jndi}).  This method allows for an administrator
to deploy a \class{RuleBase} as a managed object.  Multiple
applications may use the JNDI-accessible \class{RuleBase} without
being concerned with parsing DRL documents and building a
\class{RuleBase} directly.  Additionally, the deployed
\class{RuleBase} may be easily replaced, with the replacement being
immediately accessible to a running application.

All components of the \class{Rule} \emph{must} be serializable 
in order for the constructed \class{RuleBase} to be deployed against
a JNDI datastore.  The various semantic component interfaces and
rule-assembly classes have already been marked as
\indexClass{Serializable}.  The class definition files must be
in the application's classpath in order to deserialize the object
structure once retrieved from the directory.  Failure to have
the classes available will result in an exception being thrown.

\begin{figure}
\begin{javaCodelisting}
DirContext ctx = new InitialDirContext( props );
ctx.bind( "cn='MyRuleBase'" );
\end{javaCodelisting}
\caption{Deploying a \indexClass{RuleBase} to a JNDI data store}
\label{deploy.jndi}
\end{figure}

\subsection{JNDI deployment utility}

\begin{implnote}
...Not yet done
\end{implnote}

The command-line utility \texttt{name-me-please} is available to
assist in the creation of a serialized \indexClass{RuleBase} in
a JNDI directory from a DRL rule definition file (Chapter~\vref{drl}).

\begin{javaCodelisting}
\texttt{name-me-please} -Dprop=value --classpath xx:yy:zz myRules.drl 'cn=MyRules'
\end{javaCodelisting}

\section{Serializing a \indexClass{RuleBase} to a file}

A \indexClass{RuleBase} can be serialized in full to a file
on disk (Figure~\vref{serialize.file}).  This method allows for a rule-base to be ``frozen'' prior
to deployment.  An application merely has to unserialize the
file to use the contained \indexClass{RuleBase}.

\begin{figure}
\begin{javaCodelisting}
File ruleBaseFile = new File( pathToFile );

FileOutputStream   fileOut = new FileOutputStream( ruleBaseFile );
ObjectOutputStream objOut  = new ObjectOutputStream( fileOut );

objOut.writeObject( ruleBase );
\end{javaCodelisting}
\caption{Serializing a \indexClass{RuleBase} to a file}
\label{serialize.file}
\end{figure}

\subsection{Serialization utility}

\begin{implnote}
...Not yet done
\end{implnote}

The command-line utility \texttt{name-me-please} is available to
assist in the creation of a serialized \indexClass{RuleBase} from 
a DRL rule definition file (Chapter~\vref{drl}).

\begin{javaCodelisting}
\texttt{name-me-please} --classpath xx:yy:zz myRules.drl myRules.ser
\end{javaCodelisting}

