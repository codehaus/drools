\chapter{Java Semantic Module}

\section{Introduction}

The Java semantic module defines addition tags which may be used
within a DRL document (Chapter~\vref{drl}).  The tags provided by
this module allow for semantics based upon the Java programming
language.  Object types may be determined using Java classes while
conditions and extractors are formulated in terms of Java expressions.
A consequence may be an arbitrary block of Java statements.

\section{Java semantic module namespace URI}

That tags for the Java semantic module are defined within the
XML namespace \texttt{http://drools.org/semantics/java}.  In order
to use the tags of the Java semantic module, this namespace should
be bound to a prefix with the DRL document.  Typically this is done
on the root \tag{rule-set} tag (Figure~\vref{java.uri}).

\begin{figure}
\begin{tagExample}
<rule-set xmlns="http://drools.org/rules"
          {\color{black}xmlns:java="http://drools.org/semantics/java"}>
    ...
</rule-set>
\end{tagExample}
\caption{Binding of Java semantic module namespace URI to a prefix}
\label{java.uri}
\end{figure}

\section{Java Semantic Module tags}

\subsection{\indexTag{class}}
\label{java.class}

\begin{tagDesc}{class}
\noattrs
\tagContent{Name of the java class.}
\end{tagDesc}

\begin{figure}
\begin{tagExample}
<java:class>
\end{tagExample}
\caption{Example of the \indexTag{class} tag}
\label{java.class}
\end{figure}

\subsection{\indexTag{condition}}
\label{java.condition}

\subsection{\indexTag{extractor}}
\label{java.extractor}

\subsection{\indexTag{consequence}}
\label{java.consequence}

